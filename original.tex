%%% Notice: This file contains a large number of \verb's 
%%%         or verbatim environments in order to display command names
%%%         or examples.  But the use of \verb/verbatim is *not* recommended. 
%%% ver.7 2018/05/15 
\documentclass{pasj01}
%\draft 
\Received{$\langle$reception date$\rangle$}
\Accepted{$\langle$acception date$\rangle$}
\Published{$\langle$publication date$\rangle$}
%% \SetRunningHead{Astronomical Society of Japan}{Usage of \texttt{pasj00.cls}}
%% \SetRunningHead{Astronomical Society of Japan}{Usage of \texttt{pasj00.cls}}


\begin{document}

\title{The Impact of Relativistic Effects on the Analysis of Magnetar Hot Spots}
\author{Chushu \textsc{Qu}\altaffilmark{1}}%
\email{sojo@g.ecc.u-tokyo.ac.jp}

\author{Yudai \textsc{Suwa}\altaffilmark{1,2}}%

\author{Teruaki \textsc{Enoto}\altaffilmark{3}}%

\author{friends}

\altaffiltext{1}{Department of Earth Science and Astronomy, The University of Tokyo, Tokyo 153-8902, Japan}
\altaffiltext{2}{Center for Gravitational Physics and Quantum Information, Yukawa Institute for Theoretical Physics, Kyoto University, Kyoto 606-8502, Japan}
\altaffiltext{3}{Department of Physics, Kyoto University, Kyoto 606-8502, Japan}


\KeyWords{magnetar${}_1$ --- light-bending${}_2$ --- \dots --- key word${}_n$}

\maketitle

\begin{abstract}
In this document (\texttt{pasj.tex}), we provide a brief explanation about \texttt{pasj01.cls}, 
the current version of PASJ's document class for authors. 
The class file, \texttt{pasj01.cls}, is prepared so that authors can typeset/preview 
articles for PASJ under the \textit{standard} {\LaTeXe} system.
Note that it is assumed that authors are used to writing documents in 
\LaTeX{} style; that is, this manual shows only the differences of 
functions provided by \texttt{pasj01.cls} and those in the \textit{standard} \LaTeXe{}.  
Here, we use the phrase ``standard \LaTeX'' for ``\LaTeXe{} without 
any optional package.''  
The old system, {\LaTeX2.09},  is no longer supported.   
\end{abstract}

\section{Introduction}
Magnetar, the highly magnetized neutron star, 

\section{Method}


\section{Result}


\section{Summary}



\end{document}

