%%%%%%%%%%%%%%%%%%%%%%%%%%%%%%%%%%%%%%%%%%%%%%%%%%%%%%%%%%%%%%%%%%%%%%%%%
%%% PASJ LaTeX template for draft(body) <2024/07/01> 
%%% 
%%% IMPORTANT NOTICE FOR AUTHORS 
%%%  1. Do NOT use \def/\renewcommand. 
%%%  2. Do NOT redefine commands provided by PASJ02.cls.   
%%%  3. LETTER article must NOT exceed ``six pages'' in length in PASJ's publication layout format. 
%%%    Do NOT change the default font setting of pasj02.cls to avoid obtaining an inaccurate page estimation. 
%%%  4. ``\draft'' creates single column and double spaces format. 
%%%     
%%% Instructions to authors: https://academic.oup.com/pasj/pages/General_Instructions
%%% Author's guide (in Japanese): https://www.asj.or.jp/pasj/guide/ 
%%%%%%%%%%%%%%%%%%%%%%%%%%%%%%%%%%%%%%%%%%%%%%%%%%%%%%%%%%%%%%%%%%%%%%%%%
\documentclass[]{pasj02} 
%\draft
\usepackage[switch,mathlines]{lineno} % add line number to manuscript
%\usepackage{natbib} 

\jyear{2024}
\Received{}%{yyyy/mm/dd}
\Accepted{}%{yyyy/mm/dd}
%\Published{yyyy/mm/dd}

%\graphicspath{{./}{figures/}} 

\begin{document} 

\title{ Title of Your Paper }

%%% begin:list of authors
% Do NOT capitalize all letters in "textsc".
\author{
 A-Firstname \textsc{A-Familyname},\altaffilmark{1}\altemailmark\orcid{0000-0000-0000-0000} \email{aaaaa@xxx.xxx.xx.xx} 
 B-Firstname \textsc{B-Familyname},\altaffilmark{2}$^{,\dag}$\orcid{0000-0000-0000-0000}
 C-Firstname \textsc{C-Familyname},\altaffilmark{3}\altemailmark \email{ccccc@xxx.xxx.xx.xx}
 and 
 D-Firstname \textsc{D-Familyname}\altaffilmark{2}\altemailmark\orcid{0000-0000-0000-0000} \email{ddddd@xxx.xxx.xx.xx}
}
\altaffiltext{1}{A-Address of Institute}
\altaffiltext{2}{B-Address of Institute}
\altaffiltext{3}{C-Address of Institute}

\footnotetext[$\dag$]{Present address: ....}

%%% end:list of authors

%% !!! Select 3 to 5 words from PASJ's key words !!! 
%% List of Key Words: https://academic.oup.com/pasj/pages/Pasj_Keywords 
%% "\KeyWords{ }" always has to be placed before ``\maketitle'' 
\KeyWords{xxxx: xxxx --- ......}  

\maketitle

\begin{abstract}
Please read ``IMPORTANT NOTICE'' carefully before preparing a manuscript.  
\end{abstract}

\pagewiselinenumbers 

\section{Introduction}

\noindent IMPORTANT NOTICE\\
1. Manuscript for submission must be in the same format as a published papers. \\
2. Line numbers should be added to the manuscript. \\
3. Do NOT use ``\verb|\def|, \verb|\renewcommand|''.\\
4. Do NOT redefine commands provided by pasj02.cls.  


\section{Section 2}\label{sec:2}

 Use \verb/\timeform{ }/ for celestial coordinates as shown in the example below.

 (RA, Dec)$_{\rm J2000.0}$ = (\timeform{1h23m45.67s}, \timeform{6D54'32.1''})
 

\subsection{Subsection}\label{ssec:21}

In the theory \citep{key-2} .......... 
\citet{key-5} found .......... 


\subsubsection{Subsubsection}\label{sssec:211}

The normal order of delimiters is parentheses ( ), brackets [ ], braces \{ \}, as follows:
\begin{eqnarray}
 \alpha = \left\{ [10^2 F(\beta)] \left[ \frac{3h}{1-(1+z)} \right]^2  \right\} . 
\end{eqnarray}


\section{Section3}\label{sec:3}

Figure captions should always begin with a declarative title, followed by a brief description (see figure \ref{fig:sample}). 
In each table, a declarative title should be given in \verb/\tbl{ }/ or \verb/\caption{ }/. 
Any notes applying to the table and specific parts appear immediately below the table with symbols.   

%%%%%%%%%%%%%%%%%%%%%%%%%%%%%%%%%%%%%%% 

\begin{figure}
 \begin{center}
  %\includegraphics[width=8cm]{fig1.eps} 
 \end{center}
\caption{ Declarative title of figure. Give a brief description to understand the figure.  
 {Alt text: Line graph... } 
}\label{fig:sample}
\end{figure}
% See the instraction below for "Alt text"
% https://academic.oup.com/pasj/pages/General_Instructions#Figures%20and%20Illustrations

\begin{table}
  \tbl{First tabular.\footnotemark[$*$] }{%
  \begin{tabular}{cccc}
      \hline
      Name & Value1 & Value2 & Value3\footnotemark[$\dag$]  \\ 
      \hline
      aaa & bbb & ccc & ddd \\
      eee & fff & ggg & hhh \\
      ....\\
      \hline
    \end{tabular}}\label{tab:first}
\begin{tabnote}
\footnotemark[$*$] Brief explanation of this table.  \\ 
\footnotemark[$\dag$] Explanation of value 3. 
%\footnotemark[$\ddag$]  ... \\ 
%\footnotemark[$\S$]  ... \\ 
%\footnotemark[$\|$]  ... \\
%\footnotemark[$\sharp$]  ... \\  
%\footnotemark[$**$]  ... \\ 
%\footnotemark[$\dag\dag$]  ... \\ 
\end{tabnote}
\end{table}

\begin{longtable}{cccc}
  \caption{Sample of ``longtable.'' }\label{tab:LTsample}  
\hline\noalign{\vskip3pt} 
  Name & Value1 & Value2 & Value3\footnotemark[$*$] \\   [2pt] 
\hline\noalign{\vskip3pt} 
\endfirsthead      
\hline\noalign{\vskip3pt} 
  Name & Value1 & Value2 & Value3 \\  [2pt] 
\hline\noalign{\vskip3pt} 
\endhead
\hline\noalign{\vskip3pt} 
\endfoot
\hline\noalign{\vskip3pt} 
\multicolumn{2}{@{}l@{}}{\hbox to0pt{\parbox{160mm}{\footnotesize
\hangindent6pt\noindent
\hbox to6pt{\footnotemark[$*$]\hss}\unskip% 
  Brief explanation of value3. 
}\hss}} 
\endlastfoot 
  aaaaa & bbbbb & ccccc & ddddd \\
  ...... & ..... & ..... & ..... \\
  ...... & ..... & ..... & ..... \\
  ...... & ..... & ..... & ..... \\ 
  wwwww & xxxxx & yyyyy & zzzzz \\
\end{longtable}

%%%%%%%%%%%%%%%%%%%%%%%%%%%%%%%%%%%%%%%


\section*{Supplementary data} 

The following supplementary data is available at PASJ online.

E-table 1  


\begin{ack}
Acknowledgement should be placed at end of main text.
(NOT after the Appendix.)
\end{ack}

\section*{Funding}
 This research was supported by ...

\section*{Data availability} 
 The data underlying this article are available ...  
% Sample Data Availability Statements 
% https://academic.oup.com/pages/open-research/research-data#Data%20Availability%20Statements


\appendix %%%%%%%%%%%%%%%%%%%%%%%%%%%%%%%%%%%%%%%%%%%%%%%%%%%%%%%%
\section*{Case of single paragraph}
 No section number is necessary. Add ``*'' after \verb/\section/.

%%%% 
\section{Case of two or more paragraphs}

 Text of appendix

\section{Case of two or more paragraphs}

 Text of appendix


% Any journal's BST file (e.g., apj.bst) can be used as PASJ's BST is unavailable.    
% \bibliographystyle{****}
% \bibliography{****}
\begin{thebibliography}{}
% Journals(e.g. A\&A,ApJ,AJ,NMRAS,PASP ...)
% Authors, Year, Journal, Vol#, Page# 
\bibitem[Aauthor et al.(2001)]{key-1}
  Aauthor, A., Bauthor, B., \& Cauthor, C.\ 2001, PASJ, 53, 000 
\bibitem[Aauthor \& Bauthor(2003a)]{key-2}
  Aauthor, A., \& Bauthor, B.\ 2003a, PASJ, 55, 000 
\bibitem[Aauthor \& Bauthor(2003b)]{key-3}
  Aauthor, A., \& Bauthor, B.\ 2003b, PASJ, 55, 000 
\bibitem[Aauthor, Cauthor, and Dauthor(2000)]{key-3}
  Aauthor, A., Cauthor, C., \& Dauthor, D.\ 2000, PASJ, vol, page   
% Books
\bibitem[Aauthor \& Eauthor(2003b)]{key-4}
  Aauthor, A., \& Euthor, E.\ 2003b, Name of Book (Tokyo: Publisher) ch.0    
% Editorial Books
\bibitem[Dauthor(2001)]{key-5}
  Dauthor A.~A.\ 2001, in Name of Book, ed.\  D.~Editor (Tokyo: Publisher), 00 
\end{thebibliography}

\end{document}

=== Frequently used abbreviation of journal names === 
\aj         AJ
\araa       ARA\&A
\apj        ApJ
\apjl       ApJL
\apjs       ApJS 
\apss       Ap\&SS
\aap        A\&A
\aapr       A\&AR
\aaps       A\&AS 
\baas       BAAS
\icarus     ICARUS 
\mnras      MNRAS 
\prd        Phys.\ Rev.\ D 
\prl        Phys.\ Rev.\ Lett.
\pasp       PASP
\pasj       PASJ 
\solphys    Sol.\ Phys. 
\ssr        Space\ Sci.\ Rev. 
\nat        Nature
\iaucirc    IAU\ Circ. 
\gca        Geochim.\ Cosmochim.\ Acta 
\jgr        J.\ Geophys.\ Res.  
\nphysa     Nucl.\ Phys.\ A 
\procspie   Proc.\ SPIE
\aip        AIP Conf.\ Proc.
\asp        ASP Conf.\ Ser. 
=====================================================
