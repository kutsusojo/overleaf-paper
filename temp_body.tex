%%%%%%%%%%%%%%%%%%%%%%%%%%%%%%%%%%%%%%%%%%%%%%%%%%%%%%%%%%%%%%%%%%%%%%%%%
%%% PASJ LaTeX template for draft(body) <2018/05/15>
%%% 
%%% IMPORTANT NOTICE FOR AUTHORS
%%% 1. ``\draft'' creates single column and double spaces format. 
%%% 
%%% 2. If you comment out ``\draft'', the output will be double column
%%%    and single space.
%%% 
%%% 3. For cross-references, the use of \label/\ref/\cite and the 
%%%    thebibliography environment is strongly recommended. 
%%% 
%%% 4. Do NOT use \def/\renewcommand.
%%% 
%%% 5. Do NOT redefine commands provided by PASJ01.cls.  
%%% 
%%% 6. LETTER article must NOT exceed ``six pages'' in length in PASJ's 
%%%    publication layout format. 
%%%    Do NOT change the default font setting of pasj01.cls  to avoid 
%%%    obtaining an inaccurate estimation.
%%%    
%%% 
%%%%%%%%%%%%%%%%%%%%%%%%%%%%%%%%%%%%%%%%%%%%%%%%%%%%%%%%%%%%%%%%%%%%%%%%%
\documentclass[]{pasj01}
\draft

\Received{}%{yyyy/mm/dd}
\Accepted{}%{yyyy/mm/dd}
%\Published{yyyy/mm/dd}
 
%%% 
% \usepackage{mathpazo}
% \usepackage[T1]{fontenc} 
%%% 
 
\begin{document} 

\title{ 
%\LETTERLABEL %%% <-- uncomment for LETTER article  
%\REVIEWLABEL %%% <-- uncomment for REVIEW article  
Title of Your Paper }

%%% begin:list of authors
% Do NOT capitalize all letters in "textsc".
\author{A-Firstname \textsc{A-Familyname}\altaffilmark{1}%
\thanks{Example: Present Address is xxxxxxxxxx}}
\altaffiltext{1}{A-Address of Institute}
\email{aaaaa@xxx.xxx.xx.xx}

\author{B-Firstname \textsc{B-Familyname},\altaffilmark{2}}
\altaffiltext{2}{B-Address of Institute}
\email{bbbbb@xxx.xxx.xx.xx}

\author{C-Firstname \textsc{C-Familyname}\altaffilmark{3}}
\altaffiltext{3}{C-Address of Institute}
\email{ccccc@xxx.xxx.xx.xx}
%%% end:list of authors

%% `\KeyWords{}' always has to be placed before ``\maketitle'' 
%%  List of Key Words:  https://academic.oup.com/pasj/pages/Pasj_Keywords 
\KeyWords{xxxx: xxxx --- ......}  


\maketitle

\begin{abstract}
Please read ``IMPORTANT NOTICE'' carefully before preparing a manuscript. 
\end{abstract}


\section{Introduction}

\noindent IMPORTANT NOTICE\\
1. ``\verb|\draft|'' creates single column and double spaces format.\\
2. If you comment out ``\verb|\draft|'', the output will be double column
   and single space.\\
3. For cross-references, the use of ``\verb|\label|, \verb|\ref|, \verb|\cite|'' 
   and the thebibliography environment is strongly recommended. \\
4. Do NOT use ``\verb|\def|, \verb|\renewcommand|''.\\
5. Do NOT redefine commands provided by PASJ01.cls.\\


\newpage

\section{Section2}

In figure \ref{fig:sample}, ...

\begin{figure}
 \begin{center}
  %\includegraphics[width=8cm]{fig1.eps} 
 \end{center}
\caption{This is the first figure.}\label{fig:sample}
\end{figure}


\section{Section 3}

In the theory (\cite{key-1})..........

\subsection{Subsection}

In the theory (\cite{key-2})..........

\subsubsection{Subsubsection}

The resent result from ...........


\newpage

\section{Section 4}

The final ..........


%%%%%%%%%%%%%%%%%%%%%%%%%%%%%%%%%%%%%%%

\begin{table}
  \tbl{This is the first tabular.\footnotemark[$*$] }{%
  \begin{tabular}{cccc}
      \hline
      Name & Value1 & Value2 & Value3\footnotemark[$\dag$]  \\ 
      \hline
      aaa & bbb & ccc & ddd \\
      eee & fff & ggg & hhh \\
      ....\\
      \hline
    \end{tabular}}\label{tab:first}
\begin{tabnote}
\footnotemark[$*$] More detail about this table.  \\ 
\footnotemark[$\dag$] Explanation of value 3. 
%\footnotemark[$\ddag$]  ... \\ 
%\footnotemark[$\S$]  ... \\ 
%\footnotemark[$\|$]  ... \\
%\footnotemark[$\sharp$]  ... \\  
%\footnotemark[$**$]  ... \\ 
%\footnotemark[$\dag\dag$]  ... \\ 
\end{tabnote}
\end{table}

\begin{longtable}{cccc}
  \caption{Sample of ``longtable."}\label{tab:LTsample}
  \hline              
  Name & Value1 & Value2 & Value3 \\ 
\endfirsthead
  \hline
  Name & Value1 & Value2 & Value3 \\
\endhead
  \hline
\endfoot
  \hline
\endlastfoot
  \hline
  aaaaa & bbbbb & ccccc & ddddd \\
  ...... & ..... & ..... & ..... \\
  ...... & ..... & ..... & ..... \\
  ...... & ..... & ..... & ..... \\ 
  wwwww & xxxxx & yyyyy & zzzzz \\
\end{longtable}

%%%%%%%%%%%%%%%%%%%%%%%%%%%%%%%%%%%%%%%


\begin{ack}
Acknowledgement should be placed at end of main text.
(NOT after the Appendix.)
\end{ack}


\appendix 
\section*{Case of single paragraph}

\section{Case of two or more paragraphs}

\section{Case of two or more paragraphs}


%%%
% See the manual for the detail.
%%%
\begin{thebibliography}{}
% Journals(e.g. A\&A,ApJ,AJ,NMRAS,PASP ...)
% Authors, Year, Journal, Vol#, Page#
% Journal Title Abbreviation >> http://www.asj.or.jp/pasj/Jabb.html
\bibitem[Aauthor et al.(2001)]{key-1}
  Aauthor, A., Bauthor, B., Cauthor, C.\ 2001, PASJ, vol, page
\bibitem[Aauthor \& Bauthor(2003a)]{key-2}
  Aauthor, A., \& Bauthor, B.\ 2003a, PASJ, vol, page   
\bibitem[Aauthor \& Bauthor(2003b)]{key-3}
  Aauthor, A., \& Bauthor, B.\ 2003b, PASJ, vol, page  
\bibitem[Aauthor, Cauthor, and Dauthor(2000)]{key-3}
  Aauthor, A., Cauthor, C., \& Dauthor, D.\ 2000, PASJ, vol, page   
% Books
\bibitem[Aauthor \& Eauthor(2003b)]{key-3}
  Aauthor, A., \& Euthor, E.\ 2003b, Name of Book (Tokyo: Publisher) ch.0    
% Editorial Books
\bibitem[Dauthor(2001)]{key-n}
  Dauthor A.~A.\ 2001, in Name of Book,
   ed.\  D.~Editor (Tokyo: Publisher), page
\end{thebibliography}


\end{document}




=== Frequently used abbreviation of journal names === 
\aj         AJ
\araa       ARA\&A
\apj        ApJ
\apjl       ApJL
\apjs       ApJS 
\apss       Ap\&SS
\aap        A\&A
\aapr       A\&AR
\aaps       A\&AS 
\baas       BAAS
\jrasc      JRASC 
\mnras      MNRAS 
\prd        Phys.\ Rev.\ D 
\prl        Phys.\ Rev.\ Lett.
\pasp       PASP
\pasj       PASJ 
\solphys    Sol.\ Phys. 
\ssr        Space\ Sci.\ Rev. 
\nat        Nature
\iaucirc    IAU\ Circ. 
\gca        Geochim.\ Cosmochim.\ Acta 
\jgr        J.\ Geophys.\ Res. 
\memsai     Mem.\ Soc.\ Astron.\ Italiana
\nphysa     Nucl.\ Phys.\ A 
\procspie   Proc.\ SPIE
\aip        AIP Conf.\ Proc.
\asp        ASP Conf.\ Ser. 
=====================================================
